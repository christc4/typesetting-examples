\section{Part 11}

\subsection{Archives}

Archives are a traditional Unix package format. They are created by the ar
program, and they are normally named with a \texttt{.a} extension. Archives are
passed to a Unix linker with the \texttt{-l} option.

Although the \texttt{ar} program is capable of creating an archive from any
type of file, it is normally used to put object files into an archive. When
it is used in this way, it creates a symbol table for the archive. The symbol
table lists all the symbols defined by any object file in the archive, and
for each symbol indicates which object file defines it. Originally the symbol
table was created by the \texttt{ranlib} program, but these days it is always
created by ar by default (despite this, many Makefiles continue to run ranlib
unnecessarily).

When the linker sees an archive, it looks at the archive's symbol table. For
each symbol the linker checks whether it has seen an undefined reference to
that symbol without seeing a definition. If that is the case, it pulls the
object file out of the archive and includes it in the link. In other words, the
linker pulls in all the object files which defines symbols which are referenced
but not yet defined.

This operation repeats until no more symbols can be defined by the archive.
This permits object files in an archive to refer to symbols defined by other
object files in the same archive, without worrying about the order in which
they appear.

Note that the linker considers an archive in its position on the command line
relative to other object files and archives. If an object file appears after an
archive on the command line, that archive will not be used to defined symbols
referenced by the object file.

In general the linker will not include archives if they provide a definition
for a common symbol. You will recall that if the linker sees a common symbol
followed by a defined symbol with the same name, it will treat the common
symbol as an undefined reference. That will only happen if there is some other
reason to include the defined symbol in the link; the defined symbol will not
be pulled in from the archive.

There was an interesting twist for common symbols in archives on old
a.out-based SunOS systems. If the linker saw a common symbol, and then saw a
common symbol in an archive, it would not include the object file from the
archive, but it would change the size of the common symbol to the size in the
archive if that were larger than the current size. The C library relied on this
behaviour when implementing the \texttt{stdin} variable.
