\section{Part 2}

I'm back, and I'm still doing the linker technical introduction.

Shared libraries were invented as an optimization for virtual memory
systems running many processes simultaneously. People noticed that
there is a set of basic functions which appear in almost every program.
Before shared libraries, in a system which runs multiple processes
simultaneously, that meant that almost every process had a copy of
exactly the same code. This suggested that on a virtual memory system
it would be possible to arrange that code so that a single copy could
be shared by every process using it. The virtual memory system would
be used to map the single copy into the address space of each process
which needed it. This would require less physical memory to run multiple
programs, and thus yield better performance.

I believe the first implementation of shared libraries was on SVR3,
based on COFF\@. This implementation was simple, and basically assigned
each shared library a fixed portion of the virtual address space.
This did not require any significant changes to the linker. However,
requiring each shared library to reserve an appropriate portion of the
virtual address space was inconvenient.

SunOS4 introduced a more flexible version of shared libraries, which
was later picked up by SVR4. This implementation postponed some of
the operation of the linker to runtime. When the program started, it
would automatically run a limited version of the linker which would
link the program proper with the shared libraries. The version of the
linker which runs when the program starts is known as the \emph{dynamic
linker}. When it is necessary to distinguish them, I will refer to the
version of the linker which creates the program as the \emph{program
linker}. This type of shared libraries was a significant change to the
traditional program linker: it now had to build linking information
which could be used efficiently at runtime by the dynamic linker.

That is the end of the introduction. You should now understand the
basics of what a linker does. I will now turn to how it does it.

\subsection{Basic Linker Data Types}

The linker operates on a small number of basic data types:
\emph{symbols}, \emph{relocations}, and \emph{contents}. These are
defined in the input object files. Here is an overview of each of these.

A symbol is basically a name and a value. Many symbols represent static
objects in the original source code–that is, objects which exist in a
single place for the duration of the program. For example, in an object
file generated from C code, there will be a symbol for each function
and for each global and static variable. The value of such a symbol is
simply an offset into the contents. This type of symbol is known as a
\emph{defined} symbol. It's important not to confuse the value of the
symbol representing the variable \texttt{my\_global\_var} with the
value of \texttt{my\_global\_var} itself. The value of the symbol
is roughly the address of the variable: the value you would get from the
expression \texttt{\&my\_global\_var} in C.

Symbols are also used to indicate a reference to a name defined in a
different object file. Such a reference is known as an \emph{undefined}
symbol. There are other less commonly used types of symbols which I will
describe later.

During the linking process, the linker will assign an address to each
defined symbol, and will \emph{resolve} each undefined symbol by finding
a defined symbol with the same name.

A relocation is a computation to perform on the contents. Most
relocations refer to a symbol and to an offset within the contents.
Many relocations will also provide an additional operand, known as
the \emph{addend}. A simple, and commonly used, relocation is ``set
this location in the contents to the value of this symbol plus this
addend''. The types of computations that relocations do are inherently
dependent on the architecture of the processor for which the linker is
generating code. For example, RISC processors which require two or more
instructions to form a memory address will have separate relocations
to be used with each of those instructions; for example, ``set this
location in the contents to the lower 16 bits of the value of this
symbol''.

During the linking process, the linker will perform all of the
relocation computations as directed. A relocation in an object file
may refer to an undefined symbol. If the linker is unable to resolve
that symbol, it will normally issue an error (but not always: for some
symbol types or some relocation types an error may not be appropriate).
The contents are what memory should look like during the execution
of the program. Contents have a size, an array of bytes, and a type.
They contain the machine code generated by the compiler and assembler
(known as \emph{text}). They contain the values of initialized variables
(\emph{data}). They contain static unnamed data like string constants
and switch tables (read-only data or \emph{rdata}). They contain
uninitialized variables, in which case the array of bytes is generally
omitted and assumed to contain only zeroes (\emph{bss}). The compiler
and the assembler work hard to generate exactly the right contents,
but the linker really doesn't care about them except as raw data. The
linker reads the contents from each file, concatenates them all together
sorted by type, applies the relocations, and writes the result into the
executable file.

\subsection{Basic Linker Operation}

At this point we already know enough to understand the basic steps used by
every linker.

\begin{itemize}
    \item Read the input object files. Determine the length and type of
          the contents. Read the symbols.

    \item Build a symbol table containing all the symbols, linking
          undefined symbols to their definitions.

    \item Decide where all the contents should go in the output
          executable file, which means deciding where they should go in
          memory when the program runs.

    \item Read the contents data and the relocations. Apply the
          relocations to the contents. Write the result to the output
          file.

    \item Optionally write out the complete symbol table with the final
          values of the symbols.
\end{itemize}
