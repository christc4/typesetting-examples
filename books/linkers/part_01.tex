\section{Part 1}

I've been working on and off on a new linker. To my surprise, I've
discovered in talking about this that some people, even some computer
programmers, are unfamiliar with the details of the linking process.
I've decided to write some notes about linkers, with the goal of
producing an essay similar to my existing one about the GNU configure
and build system.

As I only have the time to write one thing a day, I'm going to do
this on my blog over time, and gather the final essay together later.
I believe that I may be up to five readers, and I hope y'all will
accept this digression into stuff that matters. I will return to random
philosophizing and minding other people's business soon enough.

\subsection{A Personal Introduction}

Who am I to write about linkers?

I wrote my first linker back in 1988, for the AMOS operating system
which ran on Alpha Micro systems. (If you don't understand the
following description, don't worry; all will be explained below). I
used a single global database to register all symbols. Object files were
checked into the database after they had been compiled. The link process
mainly required identifying the object file holding the main function.
Other objects files were pulled in by reference. I reverse engineered
the object file format, which was undocumented but quite simple. The
goal of all this was speed, and indeed this linker was much faster than
the system one, mainly because of the speed of the database.

I wrote my second linker in 1993 and 1994. This linker was designed and
prototyped by Steve Chamberlain while we both worked at Cygnus Support
(later Cygnus Solutions, later part of Red Hat). This was a complete
reimplementation of the BFD based linker which Steve had written a
couple of years before. The primary target was a.out and COFF\@. Again
the goal was speed, especially compared to the original BFD based
linker. On SunOS 4 this linker was almost as fast as running the cat
program on the input .o files.

The linker I am now working, called gold, on will be my third. It is
exclusively an ELF linker. Once again, the goal is speed, in this case
being faster than my second linker. That linker has been significantly
slowed down over the years by adding support for ELF and for shared
libraries. This support was patched in rather than being designed
in. Future plans for the new linker include support for incremental
linking--which is another way of increasing speed.

There is an obvious pattern here: everybody wants linkers to be faster.
This is because the job which a linker does is uninteresting. The linker
is a speed bump for a developer, a process which takes a relatively long
time but adds no real value. So why do we have linkers at all? That
brings us to our next topic.

\subsection{A Technical Introduction}

What does a linker do?

It's simple: a linker converts object files into executables and shared
libraries. Let's look at what that means. For cases where a linker is
used, the software development process consists of writing program
code in some language: e.g., C or C++ or Fortran (but typically not
Java, as Java normally works differently, using a loader rather than
a linker). A compiler translates this program code, which is human
readable text, into into another form of human readable text known as
assembly code. Assembly code is a readable form of the machine language
which the computer can execute directly. An assembler is used to turn
this assembly code into an object file. For completeness, I'll note that
some compilers include an assembler internally, and produce an object
file directly. Either way, this is where things get interesting.

In the old days, when dinosaurs roamed the data centers, many programs
were complete in themselves. In those days there was generally no
compiler–people wrote directly in assembly code--and the assembler
actually generated an executable file which the machine could execute
directly. As languages liked Fortran and Cobol started to appear, people
began to think in terms of libraries of subroutines, which meant that
there had to be some way to run the assembler at two different times,
and combine the output into a single executable file. This required the
assembler to generate a different type of output, which became known as
an object file (I have no idea where this name came from). And a new
program was required to combine different object files together into
a single executable. This new program became known as the linker (the
source of this name should be obvious).

Linkers still do the same job today. In the decades that followed, one
new feature has been added: shared libraries.
