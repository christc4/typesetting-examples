\section{Part 9}

\subsection{Symbol Versions}

A shared library provides an API. Since executables are built with a specific
set of header files and linked against a specific instance of the shared
library, it also provides an ABI. It is desirable to be able to update the
shared library independently of the executable. This permits fixing bugs in the
shared library, and it also permits the shared library and the executable to
be distributed separately. Sometimes an update to the shared library requires
changing the API, and sometimes changing the API requires changing the ABI.
When the ABI of a shared library changes, it is no longer possible to update
the shared library without updating the executable. This is unfortunate.

For example, consider the system C library and the \texttt{stat} function. When
file systems were upgraded to support 64-bit file offsets, it became necessary
to change the type of some of the fields in the \texttt{stat} struct. This
is a change in the ABI of \texttt{stat}. New versions of the system library
should provide a \texttt{stat} which returns 64-bit values. But old existing
executables call \texttt{stat} expecting 32-bit values. This could be addressed
by using complicated macros in the system header files. But there is a better
way.

The better way is symbol versions, which were introduced at Sun and extended
by the GNU tools. Every shared library may define a set of symbol versions,
and assign specific versions to each defined symbol. The versions and symbol
assignments are done by a script passed to the program linker when creating the
shared library.

When an executable or shared library A is linked against another shared library
B, and A refers to a symbol S defined in B with a specific version, the
undefined dynamic symbol reference S in A is given the version of the symbol S
in B. When the dynamic linker sees that A refers to a specific version of S, it
will link it to that specific version in B. If B later introduces a new version
of S, this will not affect A, as long as B continues to provide the old version
of S.

For example, when \texttt{stat} changes, the C library would provide two
versions of \texttt{stat}, one with the old version (e.g., LIBC\_1.0), and
one with the new version (LIBC\_2.0). The new version of \texttt{stat}
would be marked as the default–the program linker would use it to satisfy
references to \texttt{stat} in object files. Executables linked against the
old version would require the LIBC\_1.0 version of \texttt{stat}, and would
therefore continue to work. Note that it is even possible for both versions of
\texttt{stat} to be used in a single program, accessed from different shared
libraries.

As you can see, the version effectively is part of the name of the symbol. The
biggest difference is that a shared library can define a specific version which
is used to satisfy an unversioned reference.

Versions can also be used in an object file (this is a GNU extension to the
original Sun implementation). This is useful for specifying versions without
requiring a version script. When a symbol name containts the \texttt{@}
character, the string before the \texttt{@} is the name of the symbol, and
the string after the \texttt{@} is the version. If there are two consecutive
\texttt{@} characters, then this is the default version.

\subsection{Relaxation}

Generally the program linker does not change the contents other than applying
relocations. However, there are some optimizations which the program linker can
perform at link time. One of them is relaxation.

Relaxation is inherently processor specific. It consists of optimizing code
sequences which can become smaller or more efficient when final addresses are
known. The most common type of relaxation is for call instructions. A processor
like the m68k supports different PC relative call instructions: one with a
16-bit offset, and one with a 32-bit offset. When calling a function which is
within range of the 16-bit offset, it is more efficient to use the shorter
instruction. The optimization of shrinking these instructions at link time is
known as relaxation.

Relaxation is applied based on relocation entries. The linker looks for
relocations which may be relaxed, and checks whether they are in range. If they
are, the linker applies the relaxation, probably shrinking the size of the
contents. The relaxation can normally only be done when the linker recognizes
the instruction being relocated. Applying a relaxation may in turn bring other
relocations within range, so relaxation is typically done in a loop until there
are no more opportunities.

When the linker relaxes a relocation in the middle of a contents, it may need
to adjust any PC relative references which cross the point of the relaxation.
Therefore, the assembler needs to generate relocation entries for all PC
relative references. When not relaxing, these relocations may not be required,
as a PC relative reference within a single contents will be valid whereever the
contents winds up. When relaxing, though, the linker needs to look through all
the other relocations that apply to the contents, and adjust PC relatives one
where appropriate. This adjustment will simply consist of recomputing the PC
relative offset.

Of course it is also possible to apply relaxations which do not change the size
of the contents. For example, on the MIPS the position independent calling
sequence is normally to load the address of the function into the \texttt{\$25}
register and then to do an indirect call through the register. When the target
of the call is within the 18-bit range of the branch-and-call instruction, it
is normally more efficient to use branch-and-call, since then the processor
does not have to wait for the load of \texttt{\$25} to complete before starting
the call. This relaxation changes the instruction sequence without changing the
size.

More tomorrow. I apologize for the haphazard arrangement of these linker notes.
I'm just writing about ideas as I think of them, rather than being organized
about that. If I do collect these notes into an essay, I'll try to make them
more structured.
