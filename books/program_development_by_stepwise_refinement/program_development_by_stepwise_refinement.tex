\documentclass[11pt,oneside]{article}
\usepackage[
  top=7em,
  bottom=7em,
  left=10em,
  right=10em,
  heightrounded
]{geometry}

% Add a little more room between paragraphs.
\setlength{\parskip}{1em}

% Turn off section numbering.
\setcounter{secnumdepth}{0}

% The old.  School.
\frenchspacing

\usepackage{fontspec}
\setmainfont{Linux Libertine O}
\setmonofont{Inconsolatazi4}

\usepackage[autocite=superscript]{biblatex}
\bibliography{references.bib}

\usepackage[dvipsnames]{xcolor}
\usepackage{hyperref}
\hypersetup{%
    pdfauthor={Niklaus Wirth},
    pdftitle={Program Development by Stepwise Refinement},
    pdfsubject={education in programming, programming techniques, stepwise program construction},
    colorlinks=true,
    linkcolor=RoyalBlue,
    citecolor=Cerulean,
    urlcolor=Cerulean,
    linktoc=all
}

\usepackage{amsmath}
\usepackage{listings}
\lstset{%
    columns = fullflexible,
    keepspaces = true,
    mathescape = true,
    showstringspaces = false,
    basicstyle = \small\itshape,
    keywordstyle = \bfseries\color{NavyBlue}\ttfamily,
    keywordstyle = [2]\small\ttfamily,
    morekeywords = {begin, variable, procedure, repeat, until, if, then, else,%
                    end, array, integer, Boolean, PRINT, FAILURE},
    otherkeywords = {:=, [, ], ;, (, )},
    morekeywords = [2]{:=, [, ], ;, (, )}
}

\title{Program Development by Stepwise Refinement}
\author{Niklaus Wirth}
\date{April 1971}

% Compact the table of contents a little more than default.
\usepackage{tocloft}
\setlength{\cftbeforesecskip}{1em}

\begin{document}
\maketitle
\tableofcontents
\newpageThe creative activity of programming-to be distinguished from coding-is usually
taught by examples serving to exhibit certain techniques.  It is here
considered as a sequence of design decisions concerning the decomposition of
tasks into subtasks and of data into data structures.  The process of
successive refinement of specifications is illustrated by a short but
nontrivial example, from which a number of conclusions are drawn regarding the
art and the instruction of programming.

\newpage\section{Introduction}

Programming is usually taught by examples.  Experience shows that the success
of a programming course critically depends on the choice of these examples.
Unfortunately, they are too often selected with the prime intent to demonstrate
what a computer can do.  Instead, a main criterion for selection should be
their suitability to exhibit certain widely applicable \emph{techniques}.
Furthermore, examples of programs are commonly presented as finished
``products'' followed by explanations of their purpose and their linguistic
details.  But active programming consists of the design of \emph{new} programs,
rather than contemplation of old programs.  As a consequence of these teaching
methods, the student obtains the impression that programming consists mainly of
mastering a language (with all the peculiarities and intricacies so abundant in
modern PL's) and relying on one's intuition to somehow transform ideas into
finished programs.  Clearly, programming courses should teach methods of design
and construction, and the selected examples should be such that a gradual
\emph{development} can be nicely demonstrated.  

This paper deals with a single example chosen with these two purposes in mind.
Some well-known techniques are briefly demonstrated and motivated (strategy of
preselection, stepwise construction of trial solutions, introduction of
auxiliary data, recursion), and the program is gradually developed in a
sequence of \emph{refinement steps}.

In each step, one or several instructions of the given program are decomposed
into more detailed instructions. This successive decomposition or refinement of
specifications terminates when all instructions are expressed in terms of an
underlying computer or programming language, and must therefore be guided by
the facilities available on that computer or language.  The result of the
execution of a program is expressed in terms of data, and it may be necessary
to introduce further data for communication between the obtained subtasks or
instructions.  As tasks are refined, so the data may have to be refined,
decomposed, or structured, and it is natural to \emph{refine program and data
specifications in parallel}.

Every refinement step implies some design decisions.  It is important that
these decision be made explicit, and that the programmer be aware of the
underlying criteria and of the existence of alternative solutions.  The
possible solutions to a given problem emerge as the leaves of a tree, each node
representing a point of deliberation and decision.  Subtrees may be considered
as \emph{families of solutions} with certain common characteristics and
structures.  The notion of such a tree may be particularly helpful in the
situation of changing purpose and environment to which a program may sometime
have to be adapted.

A guideline in the process of stepwise refinement should be the principle to
decompose decisions as much as possible, to untangle aspects which are only
seemingly interdependent, and to defer those decisions which concern details of
representation as long as possible.  This will result in programs which are
easier to adapt to different environments (languages and computers), where
different representations may be required.

The chosen sample problem is formulated at the beginning of section 3. The
reader is strongly urged to try to find a solution by himself before embarking
on the paper which-of course-presents only one of many possible solutions.

\newpage\section{Notation}

For the description of programs, a slightly augmented \emph{Algol 60} notation
will be used.  In order to express repetition of statements in a more lucid way
than by use of labels and jumps, a statement of the form

\begin{lstlisting}
repeat $\langle$statement sequence$\rangle$
until $\langle$boolean expression$\rangle$
\end{lstlisting}

is introduced, meaning that the statement sequence is to be repeated until
the Boolean expression has obtained the value \textbf{true}.

\newpage\section{The 8-Queens Problem and an Approach to its Solution\autocite{dijkstra1968}}

\begin{quote}
Given are an 8 $\times$ 8 chessboard and 8 queens which are hostile to each
other. Find a position for each queen (a configuration) such that no queen may
be taken by any other queen (i.e.\ such that every row, column, and diagonal
contains at most one queen).
\end{quote}

This problem is characteristic for the rather frequent situation where an
analytical solution is not known, and where one has to resort to the method of
trial and error.  Typically, there exists a set $A$ of candidates for
solutions, among which one is to be selected which satisfies a certain
condition $p$. Thus a solution is characterized as an $x$ such that $(x \in A)
\wedge p(x)$.

A straightforward program to find a solution is:

\begin{lstlisting}[basicstyle=\small\normalfont]
repeat Generate the next element of $A$ call it $x$
until $p(x) \vee $(no more elements in $A$);
if $p(x)$ then $x$ = solution
\end{lstlisting}

The difficulty with this sort of problem usually is the sheer size of $A$,
which forbids an exhaustive generation of candidates on the grounds of
efficiency considerations.  In the present example, $A$ consists of $64!/(56!
\times 8!) = 2^{32}$ elements (board configurations).  Under the assumption
that generation and test of each configuration consumes 100 microseconds, it
would roughly take 7 hours to find a solution. It is obviously necessary to
invent a ``shortcut'', a method which eliminates a large number of
``obviously'' disqualified contenders.  This \emph{strategy} of
\emph{preselection} is characterized as follows: Find a representation of $p$
in the form $p = q \wedge r$.  Then let $B_r = \{x | (x \in A) \wedge r(x)\}$.
Obviously $B_r \subseteq A$.  Instead of generating elements of $A$, only
elements of $B$ are produced and tested on condition $q$ instead of $p$.
Suitable candidates for a condition $r$ are those which satisfy the following
requirements:

\begin{enumerate}
    \item $B_r$ is much smaller than $A$.
    \item Elements of $B_r$ are easily generated.
    \item Condition $q$ is easier to test than condition $p$.
\end{enumerate}

The corresponding program then is:

\begin{lstlisting}[basicstyle=\small\normalfont]
repeat Generate the next element of $B$ and call it $x$
until $q(x) \vee $(no more elements in $B$);
if $q(x)$ then $x$ = solution
\end{lstlisting}

A suitable condition $r$ in the 8-queens problem is the rule that in every
column of the board there must be exactly one queen.  Condition $q$ then merely
specifies that there be at most one queen in every row and in every diagonal,
which is evidently somewhat easier to test than $p$.  The set $B_r$,
(configurations with one queen in every column) contains ``only'' $8^8 =
2^{24}$ elements.  They are generated by restricting the movement of queens to
columns.  Thus all of the above conditions are satisfied.

Assuming again a time of 100 $\mu$s for the generation and test of a potential
solution, finding a solution would now consume only 100 seconds.  Having a
powerful computer at one's disposal, one might easily be content with this gain
in performance.  If one is less fortunate and is forced to, say, solve the
problem by hall (it, it would take 280 hours of generating and testing
configurations at the rate of one per second.  In this case it might pay to
spend some time finding further shortcuts.  Instead of applying the same method
as before, another one is advocated here which is characterized as follows:
Find a representation of trial solutions $x$ of the form $[x_1, x_2, \ldots
,x_n]$, such that every trial solution can be generated in steps which produce
$[x_1], [x_1, x_2], [x_1, x_2, \ldots , x_n]$ respectively.  The decomposition
must be such that:

\begin{enumerate}
    \item Every step (generating $x_j$) must be considerably simpler to compute
          than the entire candidate $x$.
    \item $q(x) \supset q(x_1 \ldots x_j)$ for all $j \leq n$.
\end{enumerate}

Thus a full solution can never be obtained by extending a partial trial
solution which does not satisfy the predicate $q$.  On the other hand, however,
a partial trial solution satisfying $q$ may not be extensible into a complete
solution.  This method of \emph{stepwise construction of trial solutions
therefore requires that trial solutions} failing at step $j$ may have to be
``shortened'' again in order to try different extensions.  This technique is
called \emph{backtracking} and may generally be characterized by the program:

\begin{lstlisting}
j := 1;
repeat trystep j;
    if successful then advance else regress
until (j < 1) $\vee$ (j > n)
\end{lstlisting}

In the 8-queens example, a solution can be constructed by positioning queens in
successive columns starting with column 1 and adding a queen in the next column
in each step.  Obviously, a partial configuration not satisfying the mutual
nonaggression condition may never be extended by this method into a full
solution.  Also, since during the $j$th step only $j$ queens have to be
considered and tested for mutual nonaggression, finding a partial solution at
step $j$ requires less effort of inspection than finding a complete solution
under the condition that all 8 queens are on the board all the time.  Both
stated criteria are therefore satisfies by the decomposition in which step $j$
consists of finding a safe position for the queen in the $j$th column.

The program subsequently to be developed is based on this method; it generates
and tests 876 partial configurations before finding a complete solution.
Assuming again that each generation and test (which is now more easily
accomplished than before) consumes one second, the solution is found in 15
minutes, and with the computer taking 100 $\mu$s per step, in 0.09 seconds.

\newpage\section{Development of the Program}

We now formulate the stepwise generation of partial solutions to the 8-queens
problem by the following first version of a program:

\begin{lstlisting}
variable board, pointer, safe;
considerfirstcolumn;
repeat trycolunm;
    if safe then
    begin setqueen; considernextcolumn
    end else regress
until lastcoldone $\vee$ regressouttofirstcol
\end{lstlisting}

This program is composed of a set of more primitive instructions (or
procedures) whose actions may be described as follows:

\begin{small}
\emph{considerthiscolumn}.  The problem essentially consists of inspecting the
safety of squares.  A pointer variable designates the currently inspected
square.  The column in which this square lies is called the currently inspected
column.  This procedure initializes the pointer to denote the first column.

\emph{trycolumn}.  Starting at the current square of inspection in the
currently considered column, move down the column until a safe square is found,
in which case the boolean variable \emph{safe} is set to \textbf{true}, or
until the last square is reached and is also unsafe, in which case the variable
\emph{safe} is set to \textbf{false}.

\emph{setqueen}.  A queen is positioned into the last inspected square.

\emph{considernextcolumn}.  Advance to the next column and initialize its
pointer or inspection.

\emph{regress}.  Regress to a column where it is possible to move the
positioned queen further down, and remove the queens positioned in the columns
over which regression takes place.  (Note that we may have to regress over at
most two columns. Why?)
\end{small}
\vspace{5mm}

The next step of program development was chosen to refine the descriptions of
the instructions \emph{trycolumn} and \emph{regress} as follows:

\begin{lstlisting}
procedure trycolumn;
repeat advancepointer; testsquare
until safe $\vee$ lastsquare

procedure regress;
    begin reconsiderpriorcolumn
        if $\neg$ regressouttofirstcol then
        begin removequeen;
            if lastsquare then
            begin reconsiderpriorcolumn;
                if $\neg$ regressouttofirstcol then
                    removequeen
            end
        end
    end
\end{lstlisting}

The program is expressed in terms of the instructions:

\begin{quote}
\emph{considerfirstcolumn}\\
\emph{considernextcolumn}\\
\emph{reconsiderpriorcolumn}\\
\emph{advancepointer}\\
\emph{testsquare} (sets the variable \emph{safe})\\
\emph{setqueen}\\
\emph{removequeen}\\
\end{quote}

and of the predicates:

\begin{quote}
\emph{lastsquare}\\
\emph{lastcoldone}\\
\emph{regressouttofirstcol}\\
\end{quote}

In order to refine these instructions and predicates further in the direction
of instructions and predicates available in common programming languages, it
becomes necessary to express them in terms of data representable in those
languages.  A decision on how to represent the relevant facts in terms of data
can therefore no longer be postponed.  First priority in decision making is
given to the problem of how to represent the positions of the queens and of the
square being currently inspected.

The most straightforward solution (i.e.\ the one most closely reflecting a
wooden chessboard occupied by marble pieces) is to introduce a Boolean square
matrix with $B[i, j] = \textbf{true}$ denoting that square $(i, j)$ is
occupied.  The success of an algorithm, however, depends almost always on a
suitable choice of its data representation in the light of the ease in which
this representation allows the necessary operations to be expressed.  Apart
from this, consideration regarding storage requirements may be of prime
importance (although hardly in this case).  A common difficulty in program
design lies in the unfortunate fact that at the stage where decisions about
data representations have to be made, it often is still difficult to foresee
the details of the necessary instructions operating on the data, and often
quite impossible to estimate the advantages of one possible representation over
another.  In general, it is therefore advisable to delay decisions about data
representation as long as possible (but not until it becomes obvious that no
realizable solution will suit the chosen algorithm).

In the problem presented here, it is fairly evident even at this stage that the
following choice is more suitable than a Boolean matrix in terms of simplicity
of later instructions as well as of storage economy.

$j$ is the index of the currently inspected column; $(x_j, j)$ is the
coordinate of the last inspected square; and the position of the queen in
column $k < j$ is given by the coordinate pair $(x_k, k)$ of the board.  Now
the variable declarations for pointer and board are refined into:

\begin{lstlisting}
integer j (0 $\leq$ j $\leq$ 9)
integer array x[1:8] (0 $\leq$ x$_i$ $\leq$ 8)
\end{lstlisting}

and the further refinements of some of the above instructions and predicates
are expressed as:

\begin{lstlisting}
procedure considerfirstcolumn;
    begin j := 1; x[1] := 0 end

procedure considernextcolumn;
    begin j := j $+$ 1; x[j] := 0 end

procedure reconsiderpriorcolumn; j := j - 1

procedure advancepointer;
    x[j] := x[j] $+$ 1

Boolean procedure lastsquare;
    lastsquare := x[j] = 8;

Boolean procedure lastcoldone;
    lastcoldone := j < 8

Boolean procedure regressouttofirstcol;
    regressouttofirstcol := j < 1
\end{lstlisting}

At this stage, the program is expressed in terms of the instructions:

\begin{quote}
\emph{testsquare}\\
\emph{setqueen}\\
\emph{removequeen}\\
\end{quote}

As a matter of fact, the instructions \emph{setqueen} and \emph{removequeen}
may be regarded as vacuous, if we decide that the procedure \emph{testsquare}
is to determine the value of the variable \emph{safe} solely on the grounds of
the values $x_1 \ldots x_{j-1}$ which completely represent the positions of the
$j - 1$ queens so far on the board.  But unfortunately the instruction
\emph{testsquare} is the one most frequently executed, and it is therefore the
one instruction where considerations of efficiency are not only justified but
essential for a good solution of the problem.  Evidently a version of
\emph{testsquare} expressed only in terms of $x_1 \ldots x_{j-1}$ is
inefficient at best.  It should be obvious that \emph{testsquare} is executed
far more often than \emph{setqueen} and \emph{removequeen}.  The latter
procedures are executed whenever the column ($j$) is changed (say $m$ times),
the former whenever a move to the next square is undertaken (i.e.\ $x_j$ is
changed, say $n$ times).  However, \emph{setqueen} and \emph{removequeen} are
the only procedures which affect the chessboard.  Effeciency may therefore be
gained by the method of \emph{introducing auxiliary variables} $V(x_1 \ldots
x_j)$ such that:

\begin{enumerate}
    \item Whether a square is safe can be completed more easily from $V(x)$
          than from $x$ directly (say in $u$ units of computation instead of
          $ku$ units of computation).

    \item The computation of $V(x)$ from $x$ (whenever $x$ changes) is not too
          complicated (say of $v$ units of computation).
\end{enumerate}

The introduction of $V$ is advantageous (apart from considerations of storage
economy), if

\begin{quote}
$n(k - 1)u > mu$ \qquad or \qquad $\dfrac{n}{m}(k - 1) > \dfrac{v}{u}$,
\end{quote}

i.e.\ if the gain is greater than the loss in computation units.

A most straightforward solution to obtain a simple version of \emph{testsquare}
is to introduce a Boolean matrix $B$ such that $B[i,j] = \textbf{true}$
signifies that square $(i, j)$ is not taken by another queen.  But
unfortunately its recomputation whenever a new queen is removed ($v$) is
prohibitive (why?) and will more than outweigh the gain.

The realization that the relevant condition for safety of a square is that the
square must lie neither in a row nor in a diagonal already occupied by another
queen, leads to a much more economic choice of $V$.  We introduce Boolean
arrays \verb|a|, \verb|b|, \verb|c| with the meanings:

\begin{quote}
$a_k = \textbf{true}$ : no queen is positioned in row $k$\\
$b_k = \textbf{true}$ : no queen is positioned in the /-diagonal $k$\\
$c_k = \textbf{true}$ : no queen is positioned in the \textbackslash-diagonal $k$\\
\end{quote}

The choice of the index ranges of these arrays is made in view of the fact that
squares with equal sum of their coordinates lie on the same /-diagonal, and
those with equal difference lie on the same \textbackslash-diagonal.  With row
and column indices from 1 to 8, we obtain:

\begin{lstlisting}
Boolean array a[1:8], b[2:16], c[-7:7]
\end{lstlisting}

Upon every introduction of auxiliary data, care has to he taken of their
\emph{correct initialization}.  Since our algorithm starts with an empty
chessboard, this fact must be represented by initially assigning the value
\textbf{true} to all components of the arrays \verb|a|, \verb|b|, and \verb|c|.
We can now write:

\begin{lstlisting}
procedure testsquare;
    safe := a[x[j]] $\wedge$ b[j $+$ x[j]] $\wedge$ c[j - x[j]]

procedure setqueen;
    a[x[j]] := b[j $+$ x[j]] := x[j - x[j]] := false

procedure removequeen;
    a[x[j]] := b[j $+$ x[j]] := c[j - x[j]] := true
\end{lstlisting}

The correctness of the latter procedure is based on the fact that each queen
currently on the board had been positioned on a safe square, and that all
queens positioned after the one to he removed now had already been removed.
Thus the square to be vacated becomes safe again.

A critical examination of the program obtained so far reveals that the variable
\verb|x[j]| occurs very often, and in particular at those places of the program
which are also executed most often.  Moreover, examination of \verb|x[j]|
occurs much more frequently than reassignment of values to \verb|j|.  As a
consequence, the principle of introduction of new auxiliary data can again be
applied to increase efficiency: a new variable

\begin{lstlisting}
integer i
\end{lstlisting}

is used to represent the value so far denoted by \verb|x[j]|.  Consequently
\verb|x[j] := i| must always be executed before \verb|j| is increased, and
\verb|i := x[j]| after \verb|j| is decreased.  This final step of program
development leads to the reformulation of some of the above procedures as
follows:

\begin{lstlisting}
procedure testsquare;
    safe := a[i] $\wedge$ b[i $+$ j] $\wedge$ c[i - j]

procedure setqueen;
    a[i] := b[i $+$ j] := c[i - j] := false

procedure removequeen;
    a[i] := b[i $+$ j] := c[i - j] := true

procedure considerfirstcolumn;
    begin j := 1; i := 0 end

procedure advancepointer; i := i $+$ 1;

procedure considernextcolumn;
    begin x[j] := i; j := j $+$ 1; i := 0 end

Boolean procedure lastsquare;
    lastsquare := i = 8
\end{lstlisting}

The final program, using the procedures

\begin{quote}
\emph{testsquare}\\
\emph{setqueen}\\
\emph{regress}\\
\emph{removequeen}\\
\end{quote}

and with the other procedures directly substituted, now has the form

\begin{lstlisting}
j := 1; i := 0;
repeat
    repeat i :=  i $+$ 1; testsquare
    until safe $\vee$ (i = 8);
    if safe then
    begin setqueen; x[j]  := i; j := j $+$ 1; i := 0
    end else regress
until (j > 8) $\vee$ (i < 1);
if j > 8 then PRINT(x) else FAILURE
\end{lstlisting}

It is noteworthy that this program still displays the structure of the version
designed in the first step. Naturally other, equally valid solutions can be
suggested and be developed by the same method of stepwise program refinement.
It is particularly essential to demonstrate this fact to students.  One
alternative solution was suggested to the author by E.~W.~Dijkstra.  It is
based on the view that the problem consists of a stepwise extension of the
board by one column containing a safely positioned queen, starting with a
null-board and terminating with 8 columns. The process of extending the board
is formulated as a procedure, and the natural method to obtain a complete board
is by \emph{recursion} of this procedure. It can easily be composed of the same
set of more primitive instructions which were used in the first solution.

\begin{lstlisting}
procedure Trycolumn(j);
    begin integer i; i := 0;
        repeat i := i $+$ 1; testsquare;
            if safe then
            begin setqueen; x[j] := i;
                if j < 8 then Trycolumn(j$+$1);
                if $\neg$ safe then removequeen
            end
        until safe $\vee$ (i = 8)
    end
\end{lstlisting}

The program using this procedure then is

\begin{lstlisting}
Trycolumn(1);
if safe then PRINT(x) else FAILURE
\end{lstlisting}

(Note that due to the introduction of the variable \verb|i| local to the
recursive procedure, every column has its own pointer of inspection \verb|i|.
As a consequence, the procedures

\begin{quote}
\emph{testsquare}\\
\emph{setqueen}\\
\emph{removequeen }\\
\end{quote}

must be declared locally within \verb|Trycolumn| too, because they refer to the
\verb|i| designating the scanned square in the \emph{current} column.)

\newpage\section{The Generalized 8-Queens Problem}

In the practical world of computing, it is rather uncommon that a program, once
it performs correctly and satisfactorily, remains unchanged forever.  Usually
its users discover sooner or later that their program does not deliver all the
desired results, or worse, that the results requested were not the ones really
needed.  Then either an extension or a change of the program is called for, and
it is in this case where the method of stepwise program design and systematic
structuring is most valuable and advantageous. If the structure and the program
components were well chosen, then often many of the constituent instructions
can be adopted unchanged.  Thereby the effort of redesign and reverification
may be drastically reduced.  As a matter of fact, the \emph{adaptability} of a
program to changes in its objectives (often called maintainability) and be
measured primarily in terms of the degree to which it is nearly structured.

It is the purpose of the subsequent section to demonstrate this advantage in
view of a generalization of the original 8-queens problem and its solution
through an extension of the program components introduced before.

The generalized problem is formulated as follows:

\begin{quote}
    Find \emph{all} possible configurations or 8 hostile queens on an
    8 $\times$ 8 chessboard, such that no queen may be taken by any other queen.
\end{quote}

The new problem essentially consists of two parts:

\begin{enumerate}
    \item Finding a method to generate further solutions.
    \item Determining whether all solutions were generated or not.
\end{enumerate}

It is evidently necessary to generate and test candidates for solutions in some
\emph{systematic manner}.  A common technique is to find an \emph{ordering of
candidates} and a condition to identify the last candidate. If an ordering is
found, the solutions can be mapped onto the integers. A condition limiting the
numeric values associated with the solutions then yields a criterion for
termination of the algorithm, if the chosen method generates solutions strictly
in increasing order.

It is easy to find orderings of solutions for the present problem. We choose
for convenience the mapping

\begin{equation*}
    M(x) = \displaystyle\sum{j = 1}^{8} x_j 10^{j-1}
\end{equation*}

An upper bound for possible solutions is then 
\begin{equation*}
    M(x_{max}) = 88888888
\end{equation*}

and the ``convenience'' lies in the circumstance that our earlier program
generating one solution generates the minimum solution which can be regarded as
the starting point from which to proceed to the next solution.  This is due to
the chosen method of testing squares strictly proceeding in increasing order of
$M(x)$ starting with $00000000$.  The method for generating further solutions
must now be chosen such that starting with the configuration of a given
solution, scanning proceeds in the same order of increasing $M$, until either
the next higher solution is found or the limit is reached.

\newpage\section{The Extended Program}

The technique of extending the two given programs finding a solution to the
simple 8-queens problem is based on the idea of modification of the global
structure only, and of using the same building blocks. The global structure
must be changed such that upon finding a solution the algorithm will produce an
appropriate indication-e.g.\ by printing the solution-and then proceed to find
the next solution until it is found or the limit is reached.  A simple
condition for reaching the limit is the event when the first queen is moved
beyond row 8, in which case regression out of the first column will take
place.  These deliberations lead to the following modified version of the
nonrecursive program:

\begin{lstlisting}
considerfirstcolumn;
    repeat trycolumn;
        if safe then
        begin setqueen; considernextcolumn;
            if lastcoldone then
            begin PRINT(x); regress
            end
        end else regress
    until regressouttofirstcol
\end{lstlisting}

Indication of a solution being found by printing it now occurs directly at
the level of detection, i.e.\ before leaving the repetition clause.  Then
the algorithm proceeds to find a next solution whereby a shortcut is used
by directly regressing to the prior column; since a solution places one
queen in each row, there is no point in further moving the last queen
within the eighth column.

The recursive program is extended with even greater ease following the same
considerations:

\begin{lstlisting}
procedure Trycolumn(j);
begin integer i;
    $\langle$declaralions of procedures testsquare, advancequeen,
    setqueen, removequeen, lastsquare$\rangle$
    i := 0;
    repeat advancequeen; testsquare;
        if safe then
        begin setqueen; x[j] := i;
            if $\neg$ lastcoldone then Trycolumn(i $+$ l) else PRINT(x);
            removequeen
        end
    until lastsquare
end
\end{lstlisting}

The main program starting the algorithm then consists (apart from
initialization of \verb|a|, \verb|b|, and \verb|c|) of the single statement
\verb|Trycolumn(1)|.

In concluding, it should be noted that both programs represent the same
algorithm. Both determine 92 solutions in the same order by testing squares
15720 times. This yields an average of 171 tests per solution; the maximum is
876 test for finding a next solution (the first one), and the minimum is 8.
(Both programs coded in the language Pascal were executed by a CDC 6400
computer in less than one second.)

\newpage\section{Conclusions}

The lessons which the described example was supposed to illustrate can be
summarized by the following points.

\begin{enumerate}
    \item Program construction consists of a sequence of \emph{refinement
          steps}.  In each step a given task is broken up into a number of
          suhtasks.  Each refinement in the description of a task may be
          accompanied by a refinement of the description of the data which
          constitute the means of communication between the subtasks.
          Refinement of the description of program and data structures should
          proceed in parallel.

    \item The degree of \emph{modularity} obtained in this way will determine
          the ease or difficulty with which a program can be adapted to changes
          or extensions of the purpose or changes in the environment (language,
          computer) in which it is executed.

    \item During the process of stepwise refinement, a \emph{notation} which is
          natural to the problem in hand should be used as long as possible.
          The direction in which the notation develops during the process of
          refinement is determined by the language in which the program must
          ultimately be specified, i.e.\ with which the notation ultimately
          becomes identical.  This language should therefore allow us to express
          as naturally and clearly as possible the structure of program and
          data which emerge during the design process. At the same time, it
          must give guidance in the refinement process by exhibiting those
          basic features and structuring principles which are natural to the
          machine by which programs are supposed to be executed.  It is
          remarkable that it would be difficult to find a language that would
          meet these important requirements to a lesser degree that the one
          language still used most widely in teaching programming: Fortran.

    \item Each refinement implies a number of \emph{design decisions} based
          upon a set of design criteria.  Among these criteria are efficiency,
          storage economy, clarity, and regularity of structure. Students must
          be taught to be conscious of the involved decisions and to critically
          examine and to reject solutions, sometimes even if they are correct
          as far as the result is concerned; they must learn to weigh the
          various aspects of design alternatives in the light of these
          criteria.  In particular, they must be taught to revoke earlier
          decisions, and to back up, if necessary even to the top.  Relatively
          short sample problems will often suffice to illustrate this important
          point; it is not necessary to construct an operating system for this
          purpose.

    \item The detailed elaborations on the development of even a short program
          form a long story, indicating that careful programming is not a
          trivial subject.  If this paper has helped to dispel the widespread
          belief that programming is easy as long as the programming language
          is powerful enough and the available computer is fast enough, then it
          has achieved one of its purposes.
\end{enumerate}

\emph{Acknowledgments}.  The author gratefully acknowledges the helpful and
stimulating influence of many discussions with C.~A.~R.~Hoare and
E.~W.~Dijkstra.

\nocite{dijkstra1969}
\nocite{naur1969}
\nocite{wirth1970}

\printbibliography
\end{document}
