\section{The Extended Program}

The technique of extending the two given programs finding a solution to the
simple 8-queens problem is based on the idea of modification of the global
structure only, and of using the same building blocks. The global structure
must be changed such that upon finding a solution the algorithm will produce an
appropriate indication-e.g.\ by printing the solution-and then proceed to find
the next solution until it is found or the limit is reached.  A simple
condition for reaching the limit is the event when the first queen is moved
beyond row 8, in which case regression out of the first column will take
place.  These deliberations lead to the following modified version of the
nonrecursive program:

\begin{lstlisting}
considerfirstcolumn;
    repeat trycolumn;
        if safe then
        begin setqueen; considernextcolumn;
            if lastcoldone then
            begin PRINT(x); regress
            end
        end else regress
    until regressouttofirstcol
\end{lstlisting}

Indication of a solution being found by printing it now occurs directly at
the level of detection, i.e.\ before leaving the repetition clause.  Then
the algorithm proceeds to find a next solution whereby a shortcut is used
by directly regressing to the prior column; since a solution places one
queen in each row, there is no point in further moving the last queen
within the eighth column.

The recursive program is extended with even greater ease following the same
considerations:

\begin{lstlisting}
procedure Trycolumn(j);
begin integer i;
    $\langle$declaralions of procedures testsquare, advancequeen,
    setqueen, removequeen, lastsquare$\rangle$
    i := 0;
    repeat advancequeen; testsquare;
        if safe then
        begin setqueen; x[j] := i;
            if $\neg$ lastcoldone then Trycolumn(i $+$ l) else PRINT(x);
            removequeen
        end
    until lastsquare
end
\end{lstlisting}

The main program starting the algorithm then consists (apart from
initialization of \verb|a|, \verb|b|, and \verb|c|) of the single statement
\verb|Trycolumn(1)|.

In concluding, it should be noted that both programs represent the same
algorithm. Both determine 92 solutions in the same order by testing squares
15720 times. This yields an average of 171 tests per solution; the maximum is
876 test for finding a next solution (the first one), and the minimum is 8.
(Both programs coded in the language Pascal were executed by a CDC 6400
computer in less than one second.)
