\section{The 8-Queens Problem and an Approach to its Solution\autocite{dijkstra1968}}

\begin{quote}
Given are an 8 $\times$ 8 chessboard and 8 queens which are hostile to each
other. Find a position for each queen (a configuration) such that no queen may
be taken by any other queen (i.e.\ such that every row, column, and diagonal
contains at most one queen).
\end{quote}

This problem is characteristic for the rather frequent situation where an
analytical solution is not known, and where one has to resort to the method of
trial and error.  Typically, there exists a set $A$ of candidates for
solutions, among which one is to be selected which satisfies a certain
condition $p$. Thus a solution is characterized as an $x$ such that $(x \in A)
\wedge p(x)$.

A straightforward program to find a solution is:

\begin{lstlisting}[basicstyle=\small\normalfont]
repeat Generate the next element of $A$ call it $x$
until $p(x) \vee $(no more elements in $A$);
if $p(x)$ then $x$ = solution
\end{lstlisting}

The difficulty with this sort of problem usually is the sheer size of $A$,
which forbids an exhaustive generation of candidates on the grounds of
efficiency considerations.  In the present example, $A$ consists of $64!/(56!
\times 8!) = 2^{32}$ elements (board configurations).  Under the assumption
that generation and test of each configuration consumes 100 microseconds, it
would roughly take 7 hours to find a solution. It is obviously necessary to
invent a ``shortcut'', a method which eliminates a large number of
``obviously'' disqualified contenders.  This \emph{strategy} of
\emph{preselection} is characterized as follows: Find a representation of $p$
in the form $p = q \wedge r$.  Then let $B_r = \{x | (x \in A) \wedge r(x)\}$.
Obviously $B_r \subseteq A$.  Instead of generating elements of $A$, only
elements of $B$ are produced and tested on condition $q$ instead of $p$.
Suitable candidates for a condition $r$ are those which satisfy the following
requirements:

\begin{enumerate}
    \item $B_r$ is much smaller than $A$.
    \item Elements of $B_r$ are easily generated.
    \item Condition $q$ is easier to test than condition $p$.
\end{enumerate}

The corresponding program then is:

\begin{lstlisting}[basicstyle=\small\normalfont]
repeat Generate the next element of $B$ and call it $x$
until $q(x) \vee $(no more elements in $B$);
if $q(x)$ then $x$ = solution
\end{lstlisting}

A suitable condition $r$ in the 8-queens problem is the rule that in every
column of the board there must be exactly one queen.  Condition $q$ then merely
specifies that there be at most one queen in every row and in every diagonal,
which is evidently somewhat easier to test than $p$.  The set $B_r$,
(configurations with one queen in every column) contains ``only'' $8^8 =
2^{24}$ elements.  They are generated by restricting the movement of queens to
columns.  Thus all of the above conditions are satisfied.

Assuming again a time of 100 $\mu$s for the generation and test of a potential
solution, finding a solution would now consume only 100 seconds.  Having a
powerful computer at one's disposal, one might easily be content with this gain
in performance.  If one is less fortunate and is forced to, say, solve the
problem by hall (it, it would take 280 hours of generating and testing
configurations at the rate of one per second.  In this case it might pay to
spend some time finding further shortcuts.  Instead of applying the same method
as before, another one is advocated here which is characterized as follows:
Find a representation of trial solutions $x$ of the form $[x_1, x_2, \ldots
,x_n]$, such that every trial solution can be generated in steps which produce
$[x_1], [x_1, x_2], [x_1, x_2, \ldots , x_n]$ respectively.  The decomposition
must be such that:

\begin{enumerate}
    \item Every step (generating $x_j$) must be considerably simpler to compute
          than the entire candidate $x$.
    \item $q(x) \supset q(x_1 \ldots x_j)$ for all $j \leq n$.
\end{enumerate}

Thus a full solution can never be obtained by extending a partial trial
solution which does not satisfy the predicate $q$.  On the other hand, however,
a partial trial solution satisfying $q$ may not be extensible into a complete
solution.  This method of \emph{stepwise construction of trial solutions
therefore requires that trial solutions} failing at step $j$ may have to be
``shortened'' again in order to try different extensions.  This technique is
called \emph{backtracking} and may generally be characterized by the program:

\begin{lstlisting}
j := 1;
repeat trystep j;
    if successful then advance else regress
until (j < 1) $\vee$ (j > n)
\end{lstlisting}

In the 8-queens example, a solution can be constructed by positioning queens in
successive columns starting with column 1 and adding a queen in the next column
in each step.  Obviously, a partial configuration not satisfying the mutual
nonaggression condition may never be extended by this method into a full
solution.  Also, since during the $j$th step only $j$ queens have to be
considered and tested for mutual nonaggression, finding a partial solution at
step $j$ requires less effort of inspection than finding a complete solution
under the condition that all 8 queens are on the board all the time.  Both
stated criteria are therefore satisfies by the decomposition in which step $j$
consists of finding a safe position for the queen in the $j$th column.

The program subsequently to be developed is based on this method; it generates
and tests 876 partial configurations before finding a complete solution.
Assuming again that each generation and test (which is now more easily
accomplished than before) consumes one second, the solution is found in 15
minutes, and with the computer taking 100 $\mu$s per step, in 0.09 seconds.
